\documentclass[a4paper,10pt,oneside]{article}
\setlength{\columnsep}{15pt}    %兩欄模式的間距
\setlength{\columnseprule}{0pt}

\usepackage[landscape]{geometry}
\usepackage{amsthm}								%定義,例題
\usepackage{amssymb}
\usepackage{fontspec}								%設定字體
\usepackage{color}
\usepackage[x11names]{xcolor}
\usepackage{xeCJK}								%xeCJK
\usepackage{listings}								%顯示code用的
%\usepackage[Glenn]{fncychap}						%排版,頁面模板
\usepackage{fancyhdr}								%設定頁首頁尾
\usepackage{graphicx}								%Graphic
\usepackage{enumerate}
\usepackage{titlesec}
\usepackage{amsmath}
\usepackage{pdfpages}
\usepackage{multicol}
\usepackage{fancyhdr}
%\usepackage[T1]{fontenc}
\usepackage{amsmath, courier, listings, fancyhdr, graphicx}

%\topmargin=0pt
%\headsep=5pt
\textheight=530pt
%\footskip=0pt
\voffset=-20pt
\textwidth=800pt
%\marginparsep=0pt
%\marginparwidth=0pt
%\marginparpush=0pt
%\oddsidemargin=0pt
%\evensidemargin=0pt
\hoffset=-100pt

%\setmainfont{Consolas}				%主要字型
\setCJKmainfont{msjh.ttc}			%中文字型
%\setmainfont{Linux Libertine G}
\setmonofont{consola.ttf}
%\setmainfont{sourcecodepro}
\XeTeXlinebreaklocale "zh"						%中文自動換行
\XeTeXlinebreakskip = 0pt plus 1pt				%設定段落之間的距離
\setcounter{secnumdepth}{3}						%目錄顯示第三層

\makeatletter
\lst@CCPutMacro\lst@ProcessOther {"2D}{\lst@ttfamily{-{}}{-{}}}
\@empty\z@\@empty
\makeatother
\lstset{											% Code顯示
language=C++,										% the language of the code
basicstyle=\scriptsize\ttfamily, 						% the size of the fonts that are used for the code
numbers=left,										% where to put the line-numbers
numberstyle=\tiny,						% the size of the fonts that are used for the line-numbers
stepnumber=1,										% the step between two line-numbers. If it's 1, each line  will be numbered
numbersep=5pt,										% how far the line-numbers are from the code
backgroundcolor=\color{white},					% choose the background color. You must add \usepackage{color}
showspaces=false,									% show spaces adding particular underscores
showstringspaces=false,							% underline spaces within strings
showtabs=false,									% show tabs within strings adding particular underscores
frame=false,											% adds a frame around the code
tabsize=2,											% sets default tabsize to 2 spaces
captionpos=b,										% sets the caption-position to bottom
breaklines=true,									% sets automatic line breaking
breakatwhitespace=false,							% sets if automatic breaks should only happen at whitespace
escapeinside={\%*}{*)},							% if you want to add a comment within your code
morekeywords={*},									% if you want to add more keywords to the set
keywordstyle=\bfseries\color{Blue1},
commentstyle=\itshape\color{Red4},
stringstyle=\itshape\color{Green4},
}


\newcommand{\includecpp}[2]{
  \subsection{#1}
    \lstinputlisting{#2}
}

\newcommand{\includetex}[2]{
  \subsection{#1}
    \input{#2}
}


\begin{document}
  \begin{multicols}{4}
  \pagestyle{fancy}
  
  \fancyfoot{}
  \fancyhead[L]{National Tsing Hua University - Angry Crow Takes Flight!}
  \fancyhead[R]{\thepage}
  
  \renewcommand{\headrulewidth}{0.4pt}
  \renewcommand{\contentsname}{Contents}

   
  \scriptsize
  \section{Data Structure}
  \includecpp{Segment Tree}{./Data Structure/Segment Tree.cpp}
  \includecpp{Treap}{./Data Structure/Treap.cpp}
\section{Flow}
  \includecpp{Dinic}{./Flow/Dinic.cpp}
  \includecpp{mcmf}{./Flow/mcmf.cpp}
\section{Graphs}
  \includecpp{BCC\_Edge}{./Graphs/BCC_Edge.cpp}
  \includecpp{BCC\_Vertex}{./Graphs/BCC_Vertex.cpp}
  \includecpp{dijkstra}{./Graphs/dijkstra.cpp}
  \includecpp{SCC\_korasaju}{./Graphs/SCC_korasaju.cpp}
  \includecpp{SCC\_tarjan}{./Graphs/SCC_tarjan.cpp}
\section{Number Theory}
  \includecpp{FFT}{./Number Theory/FFT.cpp}
  \includecpp{Linear\_Sieve}{./Number Theory/Linear_Sieve.cpp}
  \includecpp{NTT}{./Number Theory/NTT.cpp}
\section{String}
  \includecpp{AC\_Automaton}{./String/AC_Automaton.cpp}
  \includecpp{KMP}{./String/KMP.cpp}
  \includecpp{Suffix\_Array}{./String/Suffix_Array.cpp}
  \includecpp{ZValue}{./String/ZValue.cpp}
\section{Sublime-Text}
  \includecpp{SublimeBuild}{./Sublime-Text/SublimeBuild.cpp}

  \clearpage
  \end{multicols}
  \newpage
  \begin{multicols}{4}
  \enlargethispage*{\baselineskip}
  \begin{center}
    \Huge\textsc{ACM ICPC Team Reference - Angry Crow Takes Flight!}
    \vspace{0.35cm}    
  \end{center}
  \tableofcontents
  \end{multicols}
  \clearpage
\end{document}
